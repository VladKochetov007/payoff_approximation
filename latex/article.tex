\documentclass[12pt]{article}
\usepackage[utf8]{inputenc}
\usepackage{amsmath}
\usepackage{amsfonts}
\usepackage{amssymb}
\usepackage{graphicx}
\usepackage{natbib}
\usepackage{hyperref}
\usepackage{setspace}
\usepackage{pgfplots}
\pgfplotsset{compat=1.18}
\onehalfspacing
\DeclareMathOperator*{\argmin}{arg\,min}
\DeclareMathOperator*{\argmax}{arg\,max}
\usepackage{amsmath, amssymb, amsthm}

% Define keywords command
\providecommand{\keywords}[1]{\textbf{Keywords:} #1}

% Remove page number from the first page
\usepackage{titling}
\usepackage{geometry}

% Adjust vertical spacing on title page
\pretitle{\vspace*{2cm}\begin{center}}
\posttitle{\end{center}\vspace{0.5cm}}
\preauthor{\begin{center}}
\postauthor{\end{center}\vspace{0.5cm}}
\predate{\begin{center}}
\postdate{\end{center}\vspace{1cm}}

\title{\rule{\textwidth}{4pt}\\[1em]
\textbf{From Distribution Forecasts to Vanilla Portfolios: \\[0.2cm]
Regularized Payoff Approximation in Discrete Space}\\[1em]
\rule{\textwidth}{2pt}}

\author{\small \textbf{Vlad Kochetov} \\
    \small Faculty of Mechanics and Mathematics, \\
    \small Taras Shevchenko National University of Kyiv \\
    \small \texttt{vladkoch@knu.ua}
}
\date{\today}

\begin{document}

% Remove page number from the first page
\thispagestyle{empty}

\maketitle

\begin{abstract}
This paper explores the methodology for approximating complex payoff profiles using a
combination of vanilla European options. The mathematical framework is presented that
allows for the decomposition of arbitrary payoff structures into a series of call and
put options with varying strike prices. A key innovation is making the target payoff
directly proportional to the estimated probability density function of the underlying
asset price at expiration. The approach is validated through numerical examples and
practical applications in financial engineering. The results demonstrate the effectiveness
of this method in replicating complex payoffs while capturing the expected price
distribution.
\end{abstract}

\newpage

\keywords{
    \small{
        Vanilla Options, 
        Payoff Profile, 
        Financial Engineering, 
        Option Pricing, 
        Approximation, 
        Numerical Methods,
        Quantitative Finance,
        Financial Derivatives,
        Computational Finance,
        Distribution Forecasts,
        Option Trading
    }
}

\section{Introduction}
The ability to replicate complex financial instruments using simpler, more liquid
derivatives is a cornerstone of modern financial engineering. Vanilla European 
options, due to their simplicity and widespread availability, serve as ideal 
building blocks for such approximations. This paper introduces an innovative approach 
to approximating arbitrary payoff profiles using a combination of these options, where 
the target payoff is designed to be directly proportional to the estimated probability 
density function (PDF) of the underlying asset price at expiration.

The traditional approach to financial modeling often relies heavily on assumptions about
the underlying price distribution, with the normal distribution being the most common choice
due to its mathematical tractability. However, empirical evidence consistently shows that
market price distributions exhibit significant deviations from normality \citep{Pokharel2024},
particularly in the form of heavy tails and higher moments that cannot be captured by simple
parametric models. This mismatch between theoretical assumptions and market reality can lead to
systematic pricing errors and risk underestimation.

The presented approach addresses these limitations by embracing the stochastic nature of markets
and explicitly incorporating the estimated price distribution into the payoff structure. Instead 
of attempting to fit market behavior into predetermined probability distributions as in 
\citep{Kuang2023} and \citep{Li2023}, this paper proposes a framework that constructs payoff 
profiles directly proportional to the empirically observed or forecasted price distribution. 
This distribution-driven modeling approach acknowledges the complex, non-Gaussian nature of 
market price movements, which often exhibit fat tails and asymmetric behavior. By making the 
payoff proportional to the estimated PDF, the framework naturally adapts to the expected price 
dynamics, providing enhanced risk-adjusted returns when the distribution forecast is accurate.

Furthermore, the framework provides a flexible mechanism for extracting value from market
inefficiencies by aligning portfolio returns with predicted price distributions, while being 
able to adapt to changing market conditions and regime shifts that affect the shape of price
distributions.

The proposed methodology bridges the gap between theoretical option pricing models
and observed market behavior by providing a practical framework for implementing
distribution-based trading strategies. By focusing on distribution-proportional payoffs
rather than arbitrary target functions, traders can develop more robust strategies that
directly leverage their distributional views of market outcomes.

\section{Literature Review}
The theoretical foundations of option portfolio optimization were established in the work of 
\citep{Carr01012001}, which demonstrated the theoretical possibility of replicating 
arbitrary payoff functions through continuous strike spaces. While this approach 
provides elegant closed-form solutions in continuous market settings, several fundamental 
limitations affect practical implementation: the discrete availability of strike prices 
in real markets, liquidity constraints, non-negligible transaction costs scaling with 
portfolio complexity, and numerical stability challenges in finite difference implementations. 

The idea of modeling complete distributions rather than point estimates was inspired by recent 
advances in time series forecasting, particularly the work of \citep{ansari2024chronos}, which 
demonstrated the effectiveness of distribution-based predictions in capturing complex temporal 
patterns. Our innovation lies in making the target payoff directly proportional to these 
predicted distributions, creating a natural alignment between market views and portfolio returns. 
While tail risk hedging has been a fundamental concept since \citep{BlackScholes1973}, where 
options were primarily used for downside protection, our approach extends this by considering 
the entire distribution shape in portfolio construction. Unlike traditional methods that focus 
solely on tail events, we propose a comprehensive framework that models and exploits the 
complete probability distribution of asset returns by making payoffs proportional to the 
estimated PDF.

The proposed method addresses these limitations through regularization techniques and 
discrete optimization methods tailored for real-world trading environments, while 
preserving the theoretical elegance of the original framework.

\section{Methodology}
\subsection{Model Assumptions}
The proposed approximation framework relies on several key assumptions:

\begin{enumerate}
    \item \textbf{Discrete Strike Space:} The model operates in a discrete strike price space 
    $\{K_i\}_{i=1}^n$, reflecting real market conditions where options are available only 
    at specific strikes.
    
    \item \textbf{Local Approximation Region:} The approximation focuses on a finite price 
    interval around the current spot price $[S_{\min}, S_{\max}]$, where $S_{\min}$ and $S_{\max}$ 
    are chosen to ensure sufficient coverage of the relevant price region.
    
    \item \textbf{European-Style Options:} The framework utilizes European-style vanilla 
    options exclusively, avoiding the complexity of early exercise features present in 
    American options.
    
    \item \textbf{Static Replication:} The model assumes a static replication approach, 
    where the portfolio weights remain constant until maturity. Dynamic rebalancing effects 
    are not considered in the basic framework.
    
    \item \textbf{Put-Call Parity Flexibility:} The implementation supports both direct 
    use of calls and puts as separate instruments, as well as expressing puts through 
    calls and spot positions using put-call parity. This approach provides flexibility 
    in portfolio construction while maintaining mathematical equivalence.
    This flexibility enables optimization of the portfolio structure based on market 
    conditions.
    
    \item \textbf{Distribution-Proportional Payoff:} The target payoff function is set to be
    directly proportional to the estimated probability density function of the underlying
    asset price at expiration, i.e., $V(S) \propto f(S)$ where $f(S)$ is the estimated PDF.
\end{enumerate}

The numerical implementation discretizes the approximation domain into a finite set of 
evaluation points. The number of discretization points is chosen to balance computational 
efficiency with approximation accuracy.

\subsection{Mathematical Framework}
Let \( f(S) \) represent the estimated probability density function of the underlying asset
price at expiration, and let \( V(S) = k f(S) \) be the target payoff profile, where \( k \)
is a scaling constant. We construct an approximation using \( n \) call options and \( m \) 
put options with strike prices \( \{K_i\}_{i=1}^{n+m} \):

\begin{equation}
k f(S) \approx \lambda S + 
    \sum_{i=1}^{n} \alpha_i C(S, K_i) + 
    \sum_{j=1}^{m} \beta_j P(S, K_j)
\end{equation}

where \( \lambda \) represents the spot position, \( \alpha_i \) and \( \beta_j \) 
are option weights. The optimal weights are found by solving the regularized least 
squares problem with L2 (Ridge) regularization:

\begin{equation}
\min_{\boldsymbol{\theta}} 
    \int_{S_{\min}}^{S_{\max}} 
        \left[ k f(S) - \hat{V}(S;\boldsymbol{\theta}) \right]^2 dS + 
    \gamma \|\boldsymbol{\theta}\|_2^2
\end{equation}

where \( \boldsymbol{\theta} = (\lambda, \{\alpha_i\}, \{\beta_j\}) \) represents
the position weights, \( \gamma \) is the regularization parameter, and 
\( S_{\min}, S_{\max} \) define the approximation domain. The L2 norm promotes 
smoother weight distributions, providing stability to the solution.

Discretizing the integral and using matrix notation, we obtain:

\begin{equation}
\boldsymbol{\theta}^* = 
    \argmin_{\boldsymbol{\theta}} 
        \|\mathbf{A}\boldsymbol{\theta} - \mathbf{b}\|_2^2 + 
        \gamma \|\boldsymbol{\theta}\|_2^2
\end{equation}

where the design matrix \( \mathbf{A} \) contains option payoffs and spot positions
evaluated at discrete price points \( \{S_k\}_{k=1}^N \), and \( \mathbf{b} \) is
the vector of target payoff values proportional to the estimated PDF.

This quadratic optimization problem admits an analytical solution:
\begin{equation}
\boldsymbol{\theta}^* = 
    (\mathbf{A}^\top \mathbf{A} + \gamma \mathbf{I})^{-1} 
    \mathbf{A}^\top \mathbf{b}
\end{equation}

\subsection{Alternative Loss Functions}
While the standard L2 loss function provides analytical tractability, alternative 
formulations may better reflect trading objectives. Two notable generalizations are:

\begin{enumerate}
    \item \textbf{L1 Regularization (Lasso):} Replacing L2 regularization term with L1 norm:
    \begin{equation}
        L(\boldsymbol{\theta}) = 
            \int_{S_{\min}}^{S_{\max}} 
                \left[ k f(S) - \hat{V}(S;\boldsymbol{\theta}) \right]^2 dS + 
            \gamma \|\boldsymbol{\theta}\|_1
    \end{equation}
    This formulation tends to produce sparse solutions by setting some weights exactly 
    to zero, which can be beneficial when portfolio simplicity is important consideration.

    \item \textbf{Weighted Error Function:} Incorporating target function magnitude 
    into the error term:
    \begin{equation}
        L(\boldsymbol{\theta}) = 
            \int_{S_{\min}}^{S_{\max}} 
                k f(S) \Big| k f(S) - \hat{V}(S;\boldsymbol{\theta}) \Big| dS + 
            \gamma \|\boldsymbol{\theta}\|_p^p
    \end{equation}
    This formulation provides error weighting proportional to the target payoff magnitude, 
    offering economic interpretation well-aligned with trading objectives. Furthermore, it 
    naturally enhances focus on regions with significant probability mass, making it particularly 
    suitable for practical applications.
\end{enumerate}

However, these generalizations lose the analytical tractability of the L2 case, 
requiring numerical optimization methods for solution.

\subsection{Numerical Implementation}
The implementation leverages the analytical solution derived in the Mathematical Framework 
section for L2-regularized problems. For L1 regularization and alternative loss functions, 
numerical optimization methods are employed.

\begin{figure}[htbp]
\centering
\begin{tikzpicture}
\begin{axis}[
    width=0.9\textwidth,
    height=6cm,
    grid=both,
    grid style={line width=.1pt, draw=gray!10},
    major grid style={line width=.2pt,draw=gray!50},
    xlabel style={font=\tiny},
    ylabel style={font=\tiny},
    tick label style={font=\tiny},
    title style={font=\small},
    legend style={font=\tiny, at={(0.02,0.98)}, anchor=north west},
    xlabel={Underlying Asset Price},
    ylabel={Payout},
    title={Regularization Methods Comparison}
]

\addplot[thick, black] table[x index=0,y index=1] {regularization_comparison.dat};
\addlegendentry{Target Payout}

\addplot[thick, dashed, red] table[x index=0,y index=2] {regularization_comparison.dat};
\addlegendentry{L2 Regularization Approximation}

\addplot[thick, dashdotted, blue] table[x index=0,y index=3] {regularization_comparison.dat};
\addlegendentry{L1 Regularization Approximation}

\end{axis}
\end{tikzpicture}

\caption{Comparison of L1 and L2 regularization methods in approximating a PDF-proportional 
payoff profile. While both methods achieve reasonable approximation quality, L2 regularization 
provides smoother results and L1 regularization favors a sparser representation with fewer 
non-zero option weights.}
\label{fig:regularization}
\end{figure}

\begin{table}[htbp]
\centering
\caption{Performance Comparison of Different Regularization Methods}
\label{tab:method_comparison}
\begin{tabular}{cccc}
\hline
$\gamma$ & L2 MAPE (\%) & L1 MAPE (\%) & Weighted MAPE (\%) \\
\hline
0.010 & 4.95 & 4.51 & 4.95 \\
0.050 & 4.93 & 4.51 & 4.95 \\
0.100 & 4.90 & 4.97 & 4.95 \\
0.500 & 4.72 & 8.15 & 4.95 \\
1.000 & 4.59 & 10.36 & 4.95 \\
\hline
\end{tabular}
\end{table}


Table~\ref{tab:method_comparison} presents a comparative analysis of the three regularization 
methods across different regularization parameters ($\gamma$). The Mean Absolute Error (MAE) 
is used as the primary metric for comparison, providing a direct measure of approximation accuracy 
in absolute terms. Lower MAE values indicate better approximation quality.

\begin{figure}[htbp]
\centering
\begin{tikzpicture}
\begin{axis}[
    width=0.9\textwidth,
    height=6cm,
    grid=both,
    grid style={line width=.1pt, draw=gray!10},
    major grid style={line width=.2pt,draw=gray!50},
    xlabel style={font=\tiny},
    ylabel style={font=\tiny},
    tick label style={font=\tiny},
    title style={font=\small},
    legend style={font=\tiny, at={(0.02,0.98)}, anchor=north west},
    xlabel={Underlying Asset Price at Maturity},
    ylabel={Payoff},
    title={Weighted Error Method Comparison}
]

\addplot[thick, black] table[x index=0,y index=1] {weighted_loss.dat};
\addlegendentry{Target Payoff}

\addplot[thick, dotted, green!50!black] table[x index=0,y index=2] {weighted_loss.dat};
\addlegendentry{Weighted Error ($\gamma=0.10$)}

\end{axis}
\end{tikzpicture}

\caption{Performance of the weighted error method, demonstrating superior accuracy in regions 
with larger probability mass. This approach provides more economically meaningful results by 
emphasizing the approximation quality where the underlying asset price is most likely to be 
at expiration.}
\label{fig:weighted}
\end{figure}

\vspace{1em}
The results demonstrate that:
\begin{itemize}
    \item While L2 regularization provides an analytical solution and shows consistent performance 
    (MAE ~1.45), it may not be the optimal choice for practical applications.
    \item L1 regularization achieves comparable accuracy for low $\gamma$ values while promoting 
    sparser solutions, but its performance deteriorates significantly as regularization increases.
    \item The weighted error method consistently outperforms both L1 and L2 approaches, maintaining 
    superior accuracy (MAE ~1.04) across all tested regularization values. This makes it the most 
    suitable choice for real-world trading applications, where accuracy in high-probability regions 
    is crucial for portfolio performance.
\end{itemize}

Despite the mathematical elegance and analytical tractability of the L2 approach, the empirical 
results strongly suggest that the weighted error method is more appropriate for practical 
implementation. Its ability to focus on regions with significant probability mass and maintain 
consistent performance across different regularization parameters makes it the preferred choice for 
real-world trading applications.

\section{Conclusion}
This paper presents a comprehensive framework for approximating PDF-proportional payoff profiles 
using vanilla European options in discrete strike spaces. The methodology addresses several 
practical challenges in financial engineering while maintaining mathematical rigor and 
computational efficiency. The developed framework operates in discrete strike spaces, directly 
reflecting real market conditions and constraints that practitioners face in implementation.

A significant contribution of this work lies in the systematic comparison of different portfolio 
construction approaches and the novel idea of making payoffs directly proportional to estimated 
price distributions. The analysis encompasses traditional L2 regularization, which provides 
analytical tractability, L1 regularization for sparse solutions, and a novel weighted error 
optimization method. Through extensive empirical testing, we demonstrate that the weighted error 
optimization consistently outperforms conventional regularization approaches in regions with 
significant probability mass. This superior performance stems from its ability to focus the 
portfolio construction where the underlying asset price is most likely to be at expiration, 
resulting in more reliable trading strategies.

The framework's implementation guidelines strike a careful balance between computational efficiency 
and approximation accuracy. This balance is achieved through thoughtful parameter selection and 
optimization techniques, making the methodology particularly suitable for real-world applications 
where both precision and computational speed are crucial considerations.

The empirical results conclusively demonstrate that while regularization methods offer mathematical 
elegance and analytical solutions in some cases, the weighted error optimization approach provides 
markedly superior results for practical trading applications by focusing on regions with significant 
probability mass in the estimated price distribution.

Looking forward, this research opens several promising avenues for future investigation. Dynamic 
rebalancing strategies could enhance the framework's adaptability to changing market conditions. 
Integration with machine learning-based distribution forecasting might improve the accuracy of 
underlying assumptions. Furthermore, extension to multi-asset scenarios would broaden the 
methodology's applicability. The framework's inherent flexibility suggests potential applications 
beyond traditional option markets, including emerging derivative products and alternative asset 
classes, positioning it as a valuable tool for modern financial engineering.

The source code of the numerical experiments is available on 
\href{https://github.com/VladKochetov007/payoff_approximation}{GitHub}.

\bibliographystyle{apalike}
\bibliography{references}

\end{document}