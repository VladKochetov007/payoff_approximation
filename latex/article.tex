\documentclass[12pt]{article}
\usepackage[utf8]{inputenc}
\usepackage{amsmath}
\usepackage{amsfonts}
\usepackage{amssymb}
\usepackage{graphicx}
\usepackage{cite}
\usepackage{hyperref}
\usepackage{setspace}
\usepackage{pgfplots}
\pgfplotsset{compat=1.18}
\onehalfspacing
\DeclareMathOperator*{\argmin}{arg\,min}
\DeclareMathOperator*{\argmax}{arg\,max}
\usepackage{amsmath, amssymb, amsthm}

% Define keywords command
\providecommand{\keywords}[1]{\textbf{Keywords:} #1}
\title{\rule{\textwidth}{4pt}\\
\textbf{From Distribution Forecasts to Vanilla Portfolios: \\[0.2cm]
Regularized Payout Approximation in Discrete Space}\\[0.2cm]
\rule{\textwidth}{2pt}}
\author{\small \textbf{Vlad Kochetov} \\
    \small Faculty of Mechanics and Mathematics, \\
    \small Taras Shevchenko National University of Kyiv \\
    \small \texttt{vladkoch@knu.ua}
}
\date{\today}

\begin{document}

\maketitle

\begin{abstract}
This paper explores the methodology for approximating complex payout profiles using a
combination of vanilla European options. The mathematical framework is presented that
allows for the decomposition of arbitrary payout structures into a series of call and
put options with varying strike prices. The approach is validated through numerical
examples and practical applications in financial engineering. The results demonstrate
the effectiveness of this method in replicating complex payouts.
\end{abstract}

\newpage

\keywords{
    \small{
        Vanilla Options, 
        Payout Profile, 
        Financial Engineering, 
        Option Pricing, 
        Approximation, 
        Numerical Methods,
        Quantitative Finance,
        Financial Derivatives,
        Computational Finance,
        Distribution Forecasts,
        Option Trading
    }
}

\section{Introduction}
The ability to replicate complex financial instruments using simpler, more liquid
derivatives is a cornerstone of modern financial engineering. Vanilla European 
options, due to their simplicity and widespread availability, serve as ideal 
building blocks for such approximations. This paper introduces an approach to 
approximating arbitrary payout profiles using a combination of these options.

\section{Literature Review}

\section{Methodology}
\subsection{Model Assumptions}

\subsection{Mathematical Framework}
Let \( V(S) \) represent the value of a target payout profile as a function of the 
underlying asset price \( S \). We construct an approximation using \( n \) call 
options and \( m \) put options with strike prices \( \{K_i\}_{i=1}^{n+m} \):

\begin{equation}
V(S) \approx \lambda S + \sum_{i=1}^{n} \alpha_i C(S, K_i) + \sum_{j=1}^{m} \beta_j P(S, K_j)
\end{equation}

where \( \lambda \) represents the spot position, \( \alpha_i \) and \( \beta_j \) 
are option weights. The optimal weights are found by solving the regularized least 
squares problem with either L2 (Ridge) or L1 (Lasso) regularization:

\begin{equation}
\min_{\boldsymbol{\theta}} \int_{S_{\min}}^{S_{\max}} \left[ V(S) - \hat{V}(S;\boldsymbol{\theta}) \right]^2 dS + \gamma \|\boldsymbol{\theta}\|_p^p
\end{equation}

where \( \boldsymbol{\theta} = (\lambda, \{\alpha_i\}, \{\beta_j\}) \) represents
the position weights, \( \gamma \) is the regularization parameter, 
\( S_{\min}, S_{\max} \) define the approximation domain, and \( p \) determines
the type of regularization. The L2 norm (\(p=2\)) promotes smoother weight 
distributions, while the L1 norm (\(p=1\)) tends to produce sparse solutions by 
setting some weights exactly to zero, which can be beneficial when portfolio 
simplicity or transaction costs are important considerations.

Discretizing the integral and using matrix notation, we solve:

\begin{equation}
\boldsymbol{\theta}^* = \argmin_{\boldsymbol{\theta}} \|\mathbf{A}\boldsymbol{\theta} - \mathbf{b}\|_2^2 + \gamma \|\boldsymbol{\theta}\|_p^p
\end{equation}

where the design matrix \( \mathbf{A} \) contains option payoffs and spot positions
evaluated at discrete price points \( \{S_k\}_{k=1}^N \), and \( \mathbf{b} \) is
the vector of target payout values.

\subsection{Numerical Implementation}
The discrete optimization problem with L2 regularization has a closed-form solution:
\begin{equation}
\boldsymbol{\theta}^* = (\mathbf{A}^\top \mathbf{A} + \gamma \mathbf{I})^{-1} \mathbf{A}^\top \mathbf{b}
\end{equation}

where $\mathbf{A} \in \mathbb{R}^{N \times (n+m+1)}$ represents the matrix of basis 
payoffs, $\mathbf{b} \in \mathbb{R}^N$ contains target payoff values, $\gamma$ 
controls solution smoothness through regularization, and $\mathbf{I}$ denotes the 
identity matrix. Evaluation points $\{S_k\}$ are typically distributed around the 
current spot price to emphasize approximation accuracy in relevant price regions. 
The L1-regularized problem is solved using iterative optimization methods.

\begin{figure}[htbp]
\centering
\begin{tikzpicture}
\begin{axis}[
    width=0.9\textwidth,
    height=6cm,
    grid=both,
    grid style={line width=.1pt, draw=gray!10},
    major grid style={line width=.2pt,draw=gray!50},
    xlabel style={font=\tiny},
    ylabel style={font=\tiny},
    tick label style={font=\tiny},
    title style={font=\small},
    legend style={font=\tiny, at={(0.02,0.98)}, anchor=north west},
    xlabel={Underlying Asset Price at Maturity},
    ylabel={Payout},
    title={Regularization Methods Comparison}
]

\addplot[thick, black] table[x index=0,y index=1] {regularization_comparison.dat};
\addlegendentry{Target Payout}

\addplot[thick, dashed, red] table[x index=0,y index=2] {regularization_comparison.dat};
\addlegendentry{L2 Approximation($\gamma=0.10$)}

\addplot[thick, dashdotted, blue] table[x index=0,y index=3] {regularization_comparison.dat};
\addlegendentry{L1 Approximation($\gamma=0.10$)}

\end{axis}
\end{tikzpicture}

\caption{Comparison of L1 and L2 regularization methods for approximating a complex 
payout profile as a function of the underlying asset price at maturity. The target 
payout (solid black) combines linear, sinusoidal, and power-law segments. Both L2 
(dashed) and L1 (dash-dotted) regularization methods achieve good approximation 
using 9 strike prices, with L1 producing a sparser portfolio.}
\label{fig:regularization}
\end{figure}

Figure~\ref{fig:regularization} demonstrates the effectiveness of both regularization 
approaches in approximating a complex payout profile. The target function combines 
linear, sinusoidal, and power-law segments, presenting a challenging test case. Both 
methods achieve good approximation quality, with L2 regularization providing slightly 
smoother results and L1 regularization favoring a sparser representation with fewer 
non-zero option weights.

\section{Results}
\subsection{Distribution-Based Trading}
The approximation method enables novel trading strategies that exploit distributional 
forecasts rather than direct price predictions. Traders can hedge against forecasted 
distribution features, express views on higher moments of price distribution, and 
implement volatility surface arbitrage strategies through carefully constructed 
option portfolios.

The key advantage emerges when combining the approximation with distribution 
forecasts $\mathbb{P}(S_T)$:
 
\begin{equation}
\mathbb{E}[V(S_T)] = \int_{0}^{\infty} V(S) \mathbb{P}(S_T = S) dS
\end{equation}

Optimal portfolio weights $\boldsymbol{\theta}^*$ maximize the expected payout under 
forecasted distribution:

\begin{equation}
\boldsymbol{\theta}^* = \argmax_{\boldsymbol{\theta}} \mathbb{E}\left[\hat{V}(S_T;\boldsymbol{\theta})\right] - \lambda \text{Var}\left(\hat{V}(S_T;\boldsymbol{\theta})\right)
\end{equation}

where $\lambda$ represents risk aversion parameter.

\section{Conclusion}

\bibliographystyle{plain}
\bibliography{references}

\end{document} 