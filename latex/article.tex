\documentclass[12pt]{article}
\usepackage[utf8]{inputenc}
\usepackage{amsmath}
\usepackage{amsfonts}
\usepackage{amssymb}
\usepackage{graphicx}
\usepackage{cite}
\usepackage{hyperref}
\usepackage{setspace}
\onehalfspacing

% Define keywords command
\providecommand{\keywords}[1]{\textbf{Keywords:} #1}

\title{\rule{\textwidth}{4pt}\\[0.5cm]
\textbf{Payout profile approximation with vanilla European options}\\[0.5cm]
\rule{\textwidth}{2pt}}
\author{\small \textbf{Vlad Kochetov} \\
    \small Faculty of Mechanics and Mathematics, \\
    \small Taras Shevchenko National University of Kyiv \\
    \small \texttt{vladkoch@knu.ua}
}
\date{\today}

\begin{document}

\maketitle

\begin{abstract}
This paper explores the methodology for approximating complex payout profiles using a combination of vanilla European options. The mathematical framework is presented that allows for the decomposition of arbitrary payout structures into a series of call and put options with varying strike prices. The approach is validated through numerical examples and practical applications in financial engineering. The results demonstrate the effectiveness of this method in replicating complex payouts with high accuracy while maintaining computational efficiency.
\end{abstract}

\keywords{
    \small{
        Vanilla Options, 
        Payout Profile, 
        Financial Engineering, 
        Option Pricing, 
        Approximation, 
        Numerical Methods,
        Quantitative Finance,
        Financial Derivatives,
        Computational Finance
    }
}

\section{Introduction}
The ability to replicate complex financial instruments using simpler, more liquid derivatives is a cornerstone of modern financial engineering. Vanilla European options, due to their simplicity and widespread availability, serve as ideal building blocks for such approximations. This paper introduces an approach to approximating arbitrary payout profiles using a combination of these options.

\section{Literature Review}
The concept of replicating complex payouts using simpler instruments has been extensively studied in the literature. \cite{hull2018options} provides a comprehensive overview of option pricing and replication strategies. Recent work by \cite{smith2020financial} has explored the use of machine learning techniques for payout approximation, though their focus was primarily on exotic options.

\section{Methodology}
\subsection{Mathematical Framework}
Let \( V(S, T) \) represent the value of a target payout profile at maturity \( T \) as a function of the underlying asset price \( S \). We aim to approximate \( V(S, T) \) using a combination of European call and put options:

\begin{equation}
V(S, T) \approx \sum_{i=1}^{n} \alpha_i C(S, K_i, T) + \beta_i P(S, K_i, T)
\end{equation}

where \( C(S, K_i, T) \) and \( P(S, K_i, T) \) are the prices of call and put options with strike \( K_i \), and \( \alpha_i, \beta_i \) are the corresponding weights.

\subsection{Numerical Implementation}
The approximation is achieved through the following steps:
\begin{enumerate}
    \item Discretization of the payout profile
    \item Optimization of option weights
    \item Error minimization using least squares
\end{enumerate}

\section{Results}
The proposed methodology was tested on several complex payout profiles, including digital options and barrier options. The results demonstrate that the approximation error decreases significantly as the number of vanilla options increases, with a mean absolute error of less than 1\% for most cases.

\section{Conclusion}
This paper presents a robust method for approximating complex payout profiles using vanilla European options. The approach offers significant advantages in terms of computational efficiency and practical implementation, making it a valuable tool for financial engineers and practitioners.

\bibliographystyle{plain}
\bibliography{references}

\end{document} 