\documentclass[12pt]{article}
\usepackage[utf8]{inputenc}
\usepackage{amsmath}
\usepackage{amsfonts}
\usepackage{amssymb}
\usepackage{graphicx}
\usepackage{cite}
\usepackage{hyperref}
\usepackage{setspace}
\onehalfspacing
\DeclareMathOperator*{\argmin}{arg\,min}

% Define keywords command
\providecommand{\keywords}[1]{\textbf{Keywords:} #1}
\title{\rule{\textwidth}{4pt}\\
\textbf{From Distribution Forecasts to Vanilla Portfolios: \\[0.2cm]
Regularized Payout Approximation in Derivative Space}\\[0.2cm]
\rule{\textwidth}{2pt}}
\author{\small \textbf{Vlad Kochetov} \\
    \small Faculty of Mechanics and Mathematics, \\
    \small Taras Shevchenko National University of Kyiv \\
    \small \texttt{vladkoch@knu.ua}
}
\date{\today}

\begin{document}

\maketitle

\begin{abstract}
This paper explores the methodology for approximating complex payout profiles using a
combination of vanilla European options. The mathematical framework is presented that
allows for the decomposition of arbitrary payout structures into a series of call and
put options with varying strike prices. The approach is validated through numerical
examples and practical applications in financial engineering. The results demonstrate
the effectiveness of this method in replicating complex payouts.
\end{abstract}

\newpage

\keywords{
    \small{
        Vanilla Options, 
        Payout Profile, 
        Financial Engineering, 
        Option Pricing, 
        Approximation, 
        Numerical Methods,
        Quantitative Finance,
        Financial Derivatives,
        Computational Finance
    }
}

\section{Introduction}
The ability to replicate complex financial instruments using simpler, more liquid
derivatives is a cornerstone of modern financial engineering. Vanilla European options,
due to their simplicity and widespread availability, serve as ideal building blocks
for such approximations. This paper introduces an approach to approximating arbitrary
payout profiles using a combination of these options.

\section{Literature Review}
Foundational work by Black and Scholes \cite{black1973pricing} established the theoretical framework for option pricing under geometric Brownian motion assumptions. Subsequent research has extended these principles to various derivative pricing applications.

\section{Methodology}
\subsection{Model Assumptions}
The analysis assumes the underlying asset follows geometric Brownian motion with constant drift $\mu$ and volatility $\sigma$ \cite{black1973pricing}:
\begin{equation}
dS_t = \mu S_t dt + \sigma S_t dW_t
\end{equation}
where $W_t$ is a Wiener process. Option pricing utilizes the Black-Scholes framework \cite{black1973pricing} for fair value calculation, incorporating risk-neutral valuation principles.

\subsection{Mathematical Framework}
Let \( V(S, T) \) represent the value of a target payout profile at maturity \( T \)
as a function of the underlying asset price \( S \). We construct an approximation
using \( n \) call options and \( m \) put options with strike prices \( \{K_i\}_{i=1}^{n+m} \):

\begin{equation}
V(S, T) \approx \lambda S + \sum_{i=1}^{n} \alpha_i C(S, K_i, T) + \sum_{j=1}^{m} \beta_j P(S, K_j, T)
\end{equation}

where \( \lambda \) represents the spot position, \( \alpha_i \) and \( \beta_j \) are
option weights. The optimal weights are found by solving the regularized least squares
problem:

\begin{equation}
\min_{\boldsymbol{\theta}} \int_{S_{\min}}^{S_{\max}} \left[ V(S) - \hat{V}(S;\boldsymbol{\theta}) \right]^2 dS + \gamma \|\boldsymbol{\theta}\|_2^2
\end{equation}

where \( \boldsymbol{\theta} = (\lambda, \{\alpha_i\}, \{\beta_j\}) \) represents
the position weights, \( \gamma \) is the regularization parameter, and
\( S_{\min}, S_{\max} \) define the approximation domain.

Discretizing the integral and using matrix notation, we solve:

\begin{equation}
\boldsymbol{\theta}^* = \argmin_{\boldsymbol{\theta}} \|\mathbf{A}\boldsymbol{\theta} - \mathbf{b}\|_2^2 + \gamma \|\boldsymbol{\theta}\|_2^2
\end{equation}

where the design matrix \( \mathbf{A} \) contains option payoffs and spot positions
evaluated at discrete price points \( \{S_k\}_{k=1}^N \), and \( \mathbf{b} \) is
the vector of target payout values.

\subsection{Numerical Implementation}
The discrete optimization problem is solved using regularized least squares:
\begin{equation}
\boldsymbol{\theta}^* = (\mathbf{A}^\top \mathbf{A} + \gamma \mathbf{I})^{-1} \mathbf{A}^\top \mathbf{b}
\end{equation}

where $\mathbf{A} \in \mathbb{R}^{N \times (n+m+1)}$ represents the matrix of basis payoffs, $\mathbf{b} \in \mathbb{R}^N$ contains target payoff values, $\gamma$ controls solution smoothness through regularization, and $\mathbf{I}$ denotes the identity matrix for L2 regularization. Evaluation points $\{S_k\}$ are typically distributed around the current spot price to emphasize approximation accuracy in relevant price regions.

\section{Results}
\subsection{Distribution-Based Trading Strategy}
The approximation method enables novel trading strategies that exploit distributional forecasts rather than direct price predictions. Traders can hedge against forecasted distribution features, express views on higher moments of price distribution, and implement volatility surface arbitrage strategies through carefully constructed option portfolios.

The key advantage emerges when combining the approximation with distribution forecasts $\mathbb{P}(S_T)$:
 
\begin{equation}
\mathbb{E}[V(S_T)] = \int_{0}^{\infty} V(S) \mathbb{P}(S_T = S) dS
\end{equation}

Optimal portfolio weights $\boldsymbol{\theta}^*$ maximize the expected payout under forecasted distribution:

\begin{equation}
\boldsymbol{\theta}^* = \argmax_{\boldsymbol{\theta}} \mathbb{E}\left[\hat{V}(S_T;\boldsymbol{\theta})\right] - \lambda \text{Var}\left(\hat{V}(S_T;\boldsymbol{\theta})\right)
\end{equation}

where $\lambda$ represents risk aversion parameter.

\begin{figure}[htbp]
\centering
\begin{tikzpicture}
\begin{axis}[
    width=0.9\textwidth,
    height=6cm,
    grid=both,
    grid style={line width=.1pt, draw=gray!10},
    major grid style={line width=.2pt,draw=gray!50},
    xlabel style={font=\tiny},
    ylabel style={font=\tiny},
    tick label style={font=\tiny},
    title style={font=\small},
    legend style={font=\tiny, at={(0.02,0.98)}, anchor=north west},
    xlabel={Underlying Asset Price at Maturity},
    ylabel={Payout},
    title={Regularization Methods Comparison}
]

\addplot[thick, black] table[x index=0,y index=1] {regularization_comparison.dat};
\addlegendentry{Target Payout}

\addplot[thick, dashed, red] table[x index=0,y index=2] {regularization_comparison.dat};
\addlegendentry{L2 Approximation($\gamma=0.10$)}

\addplot[thick, dashdotted, blue] table[x index=0,y index=3] {regularization_comparison.dat};
\addlegendentry{L1 Approximation($\gamma=0.10$)}

\end{axis}
\end{tikzpicture}

\caption{Comparison of L1 and L2 regularization methods for payout profile approximation}
\label{fig:regularization}
\end{figure}

\section{Conclusion}

\bibliographystyle{plain}
\bibliography{references}

\end{document} 